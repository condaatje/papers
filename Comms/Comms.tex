\documentclass[12pt]{article}
\usepackage[hyperfootnotes=false]{hyperref}
\usepackage{multicol}
\usepackage{pseudocode}
\usepackage[margin=0.75in]{geometry}
\usepackage{graphicx}
\usepackage{multicol}


\thispagestyle{empty}

\begin{document}

% Was high. Was at Page 141 in Titan by Ron Chernow, reading about Rockefeller's collusive maneuvering in 1872.

% \noindent A bit of economic game theory, maybe a bit obvious:

\paragraph{Thoughts on Rational Cartelization Without Collusion} \ 

\

\noindent Let's say there's an emerging monopolizer A with price $P_a$ which naturally must be less than or equal to his costs $C_a$ pushing prices down, undercutting two other suppliers with prices $P_b$ and $P_c$. Logically, C's price is greater than or equal to B's which is in turn greater than or equal to A's.

\

\noindent A has already bumped some number of competetive suppliers off the friction cliff. The only rational cause for A to raise his price before B and C succumb to the cliff is if it is more long-term beneficial for A if C and B to continue to exist. The only situation in which this makes sense would normally be caused by illegal collusion - the end result being A B and C not competing entirely.

\

\noindent However, what if A \emph{does} raise his price, seemingly irrationally, without any communication between suppliers. Because B has not colluded, he does not know that A and C have not done so. In fact, B knows it would be irrational for A to raise his prices otherwise, and there are no regulatory/financial consequences to raising $P_b$ because B has not colluded. (B might even assume A and C will be punished and their competetiveness relative to B may actually diminish).

\

\noindent This means B will happily raise $P_b$ and be correct in doing so. But now C, having gone through a similar thought process, will be absolutely certain A and B have colluded. Up goes $P_c$.

\

\noindent C, as a member of the sadly fictional species \emph{Homo Economicus}, knows all this because he has perfect information (can go down the theory of mind rabbit hole with A and B too). Therefore, raising $P_c$ is in fact rational. This leads to effects of communication without any actually taking place.

\

\noindent I'll claim the induction is trivial for any number of would-be collusive suppliers with prices higher than or equal to $P_c$. As such, anti-collusive regulation only barring communication between parties in a perfectly rational economy is not functional.

\

\noindent This means that even if it were possible to prevent communication between nodes in a distributed economic network (such as Squire), blocking that communication would not eliminate cooperation.

\

\noindent \textbf{In short: cooperation != communication.}


\end{document}
